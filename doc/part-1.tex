\documentclass[DIN, pagenumber=false, fontsize=11pt, parskip=half]{scrartcl}

\usepackage{ngerman}
\usepackage[utf8]{inputenc}
\usepackage[T1]{fontenc}
\usepackage{textcomp}
\usepackage{xyling}

% for matlab code
% bw = blackwhite - optimized for print, otherwise source is colored
%\usepackage[framed,numbered,bw]{mcode}

% for other code
%\usepackage{listings}

\setlength{\parindent}{0em}

% set section in CM
\setkomafont{section}{\normalfont\bfseries\Large}

\newcommand{\mytitle}[1]{{\noindent\Large\textbf{#1}}}
\newcommand{\mysection}[1]{\textbf{\section*{#1}}}
\newcommand{\mysubsection}[2]{\romannumeral #1) #2}

%===================================
\begin{document}

\noindent\textbf{Foundations 2} \hfill \textbf{Heriot Watt University}\\
\hfill Calum Gilchrist\\

\mytitle{Foundations 2 Assignment: Part 1 \hfill \today}


%===================================

\section{Files}

\subsection{all.h}

Contains all the structure definitions. Due to recursive dependencies
the structures are all defined together.

\subsection{constants.h}

Constants required for determining the type of data stored in a
container. Ideally these should be stored as enum, but instead are
constant definitions.

\subsection{value.c}

The Value structure is the main storage container used. The Value stores
either an Integer, Set or a Pair inside a Container Union. The Value
also stores a type value, which is defined in \texttt{constants.h}.

\subsubsection{Methods}

\texttt{create\_empty\_value} function. This
function takes a type and creates a new Value with the type field set to
that of the type given in arguments. The function then returns the new
Value.

\texttt{value\_equality} Uses the underlying Value.type to determine the equality, using the type\_print method, defined by that structure.

\subsection{pair.c}

The Pair struct has two Value child elements: left and right. 

\subsubsection{Methods}

The \texttt{create\_pair} method creates a new Pair, with the left and right
branches assigned and returns the Value representation of this new
pair.

\texttt{pair\_equality} compares each side of the pair with the corresponding side of the second pair. Will not work for different ordered pairs.

\subsection{set.c}

The Set structure is implemented as a linked list, so every member of
the set knows about it's following member, until you get to the end of
the set. A member in the list is a node with a next element and a stored Value.

The \texttt{create\_set} function creates a new Set in
memory and returns the address inside the set\_new argument.

The \texttt{insert\_el} function adds new elements onto the end of the
set. The set is recursed through and when the end is found, a new element with the Value of key is created. 

\texttt{set\_contents\_equality} compares the contents of two sets.
Every element in first is compared with the element in the same position
in second. This means that unordered sets are not dealt with. Two sets
are not equal if the function finds any element pair that are not equal
in the respective sets, before reaching the end of the set.

\subsubsection{Set Methods}

The Set structure also has some set specific functions for manipulating
the sets.

\begin{itemize}
\item
  Union - The union appends every element of the second set onto the end
  of the first set.
\item
  Subtraction (Complements) of two sets - finds the elements in the
  first set that are also in the second set and removes them from the
  first set.
\item
  Intersection - Searches through the sets finding equal values and
  builds a new set from all the elements that are equal to one another.
\end{itemize}

\subsection{z-notation-parser.c}

The main file. Builds the sets required for the assignment and outputs them to the specification.

\end{document}
