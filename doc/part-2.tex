\documentclass[DIN, pagenumber=false, fontsize=11pt, parskip=half]{scrartcl}

\usepackage{ngerman}
\usepackage[utf8]{inputenc}
\usepackage[T1]{fontenc}
\usepackage{textcomp}
\usepackage{xyling}

\usepackage{hyperref}

% for matlab code
% bw = blackwhite - optimized for print, otherwise source is colored
%\usepackage[framed,numbered,bw]{mcode}

% for other code
%\usepackage{listings}

\setlength{\parindent}{0em}

% set section in CM
\setkomafont{section}{\normalfont\bfseries\Large}

\newcommand{\mytitle}[1]{{\noindent\Large\textbf{#1}}}
\newcommand{\mysection}[1]{\textbf{\section*{#1}}}
\newcommand{\mysubsection}[2]{\romannumeral #1) #2}

%===================================
\begin{document}

\noindent\textbf{Foundations 2} \hfill \textbf{Heriot Watt University}\\
\hfill Calum Gilchrist\\

\mytitle{Foundations 2 Assignment: Part 2 \hfill \today}


%===================================

\section{Part-2}

\texttt{part-2.c} is the main file for the second part of the assignment. The
\emph{JSON} input file was used with the
cJSON (\hyperref[cJSON]{http://sourceforge.net/projects/cjson/}) C library.
\emph{cJSON} was used as it uses a tree structure for the JSON making parsing
simpler.
The main methods of \texttt{part-2.c} are:

\begin{itemize}
    \item \texttt{parse\_operator}: This method is the outermost part of the parser. It
    takes the root node and then loops through the array of variable objects.
While doing this, it builds a global \emph{Variable} array.  \item
    \texttt{parse\_equal\_op} The first equal operation is used as allocator.
    It allocates the values in t \item \texttt{parse\_tuple\_op},
    \texttt{parse\_set\_op} build the basic structures used, the Set and the
Pair from the underlying JSON.  \item \texttt{parse\_equality\_op}: Compares
    two values, sets containg the same elements are equal. Pairs are equal if
    the elements are equal and in the same order.  \item
    \texttt{parse\_member\_op}: Checks if a value is contained in a set.
Returns 1 if the element is contained within the set and 0 if otherwise.  \item
    \texttt{parse\_base\_type} Used to check the types of operators and execute
    the method related to that type of operation, e.g. building a set for a set
    operation.
\end{itemize}

Added a Variable type for handling the variables in a reasonable manner. Store
the name, e.g. x1 and the value that the name corresponds to.

\section{Changes}

There were several changes made to the structures from Part 1. Here is a breif summary:

\begin{itemize}
    \item Changed print methods so that they can optionally be printed out to a file, rather than always standard out
    \item Made some new create statements in Pair and Value to assist in a few pointer issues.
    \item Fixed the Set Equality method to work with sets not in sequence.
\end{itemize}

\end{document}
